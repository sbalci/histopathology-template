\documentclass[]{article}
\usepackage{lmodern}
\usepackage{amssymb,amsmath}
\usepackage{ifxetex,ifluatex}
\usepackage{fixltx2e} % provides \textsubscript
\ifnum 0\ifxetex 1\fi\ifluatex 1\fi=0 % if pdftex
  \usepackage[T1]{fontenc}
  \usepackage[utf8]{inputenc}
\else % if luatex or xelatex
  \ifxetex
    \usepackage{mathspec}
  \else
    \usepackage{fontspec}
  \fi
  \defaultfontfeatures{Ligatures=TeX,Scale=MatchLowercase}
\fi
% use upquote if available, for straight quotes in verbatim environments
\IfFileExists{upquote.sty}{\usepackage{upquote}}{}
% use microtype if available
\IfFileExists{microtype.sty}{%
\usepackage{microtype}
\UseMicrotypeSet[protrusion]{basicmath} % disable protrusion for tt fonts
}{}
\usepackage[margin=1in]{geometry}
\usepackage{hyperref}
\hypersetup{unicode=true,
            pdftitle={Histopathology Research Template},
            pdfborder={0 0 0},
            breaklinks=true}
\urlstyle{same}  % don't use monospace font for urls
\usepackage{graphicx,grffile}
\makeatletter
\def\maxwidth{\ifdim\Gin@nat@width>\linewidth\linewidth\else\Gin@nat@width\fi}
\def\maxheight{\ifdim\Gin@nat@height>\textheight\textheight\else\Gin@nat@height\fi}
\makeatother
% Scale images if necessary, so that they will not overflow the page
% margins by default, and it is still possible to overwrite the defaults
% using explicit options in \includegraphics[width, height, ...]{}
\setkeys{Gin}{width=\maxwidth,height=\maxheight,keepaspectratio}
\IfFileExists{parskip.sty}{%
\usepackage{parskip}
}{% else
\setlength{\parindent}{0pt}
\setlength{\parskip}{6pt plus 2pt minus 1pt}
}
\setlength{\emergencystretch}{3em}  % prevent overfull lines
\providecommand{\tightlist}{%
  \setlength{\itemsep}{0pt}\setlength{\parskip}{0pt}}
\setcounter{secnumdepth}{5}
% Redefines (sub)paragraphs to behave more like sections
\ifx\paragraph\undefined\else
\let\oldparagraph\paragraph
\renewcommand{\paragraph}[1]{\oldparagraph{#1}\mbox{}}
\fi
\ifx\subparagraph\undefined\else
\let\oldsubparagraph\subparagraph
\renewcommand{\subparagraph}[1]{\oldsubparagraph{#1}\mbox{}}
\fi

%%% Use protect on footnotes to avoid problems with footnotes in titles
\let\rmarkdownfootnote\footnote%
\def\footnote{\protect\rmarkdownfootnote}

%%% Change title format to be more compact
\usepackage{titling}

% Create subtitle command for use in maketitle
\providecommand{\subtitle}[1]{
  \posttitle{
    \begin{center}\large#1\end{center}
    }
}

\setlength{\droptitle}{-2em}

  \title{Histopathology Research Template}
    \pretitle{\vspace{\droptitle}\centering\huge}
  \posttitle{\par}
    \author{true}
    \preauthor{\centering\large\emph}
  \postauthor{\par}
      \predate{\centering\large\emph}
  \postdate{\par}
    \date{2019-11-06}

\usepackage{pdflscape}
\newcommand{\blandscape}{\begin{landscape}}
\newcommand{\elandscape}{\end{landscape}}
\usepackage{xcolor}
\definecolor{fancyTextColor}{HTML}{4284f5}
\definecolor{fancyTextColor}{RGB}{66, 132, 245}
\definecolor{fancyTextColor}{rgb}{0.258, 0.517, 0.960}
\usepackage{afterpage}

\begin{document}
\maketitle

{
\setcounter{tocdepth}{5}
\tableofcontents
}
iris \%\textgreater\% mutate(sumVar = rowSums(.{[}1:4{]}))

iris \%\textgreater\% mutate(sumVar = rowSums(select(.,
contains(``Sepal'')))) \%\textgreater\% head

iris \%\textgreater\% mutate(sumVar = select(., contains(``Sepal''))
\%\textgreater\% rowSums()) \%\textgreater\% head

\begin{verbatim}
$index
# A tibble: 3 x 2
  var1        var2       
  <chr>       <chr>      
1 sex.factor  sex.factor2
2 age.factor  age.factor2
3 sex.factor2 age.factor2

$counts
$counts[[1]]
# A tibble: 2 x 3
  sex.factor sex.factor2     n
  <fct>      <fct>       <int>
1 Female     M             445
2 Male       F             484

$counts[[2]]
# A tibble: 3 x 3
  age.factor  age.factor2     n
  <fct>       <fct>       <int>
1 <40 years   <60 years      70
2 40-59 years <60 years     344
3 60+ years   60+ years     515

$counts[[3]]
# A tibble: 4 x 3
  sex.factor2 age.factor2     n
  <fct>       <fct>       <int>
1 M           <60 years     204
2 M           60+ years     241
3 F           <60 years     210
4 F           60+ years     274
\end{verbatim}

iRenameColumn.R

iSelectColumn.R

\begin{verbatim}
<= 22 Low
>= 23 & <= 41 Average 
>=42 High
\end{verbatim}

\hypertarget{impute}{%
\section{impute}\label{impute}}

\hypertarget{impute-continious}{%
\subsection{impute continious}\label{impute-continious}}

\hypertarget{impute-categorical}{%
\subsection{impute categorical}\label{impute-categorical}}

\hypertarget{impute-outlier}{%
\subsection{impute outlier}\label{impute-outlier}}

\hypertarget{transform}{%
\section{transform}\label{transform}}

\hypertarget{min--max}{%
\subsection{min -max}\label{min--max}}

\hypertarget{skewness}{%
\subsection{skewness}\label{skewness}}

\hypertarget{log}{%
\subsection{log}\label{log}}

\hypertarget{binning}{%
\section{binning}\label{binning}}

\hypertarget{optimal-binning}{%
\subsection{optimal binning}\label{optimal-binning}}

\hypertarget{standardize}{%
\subsection{standardize}\label{standardize}}

\hypertarget{data-transformation-report}{%
\section{data transformation report}\label{data-transformation-report}}

\hypertarget{inspectdf}{%
\section{inspectdf}\label{inspectdf}}

\pagebreak

\hypertarget{descriptive-statistics}{%
\section{Descriptive Statistics}\label{descriptive-statistics}}

\hypertarget{table-1}{%
\subsection{Table 1}\label{table-1}}

\begin{center}\rule{0.5\linewidth}{\linethickness}\end{center}

\hypertarget{categorical-variables}{%
\subsection{Categorical Variables}\label{categorical-variables}}

\hypertarget{split-group-stats-categorical}{%
\subsubsection{Split-Group Stats
Categorical}\label{split-group-stats-categorical}}

\hypertarget{grouped-categorical}{%
\subsubsection{Grouped Categorical}\label{grouped-categorical}}

\pagebreak

\begin{center}\rule{0.5\linewidth}{\linethickness}\end{center}

\hypertarget{continious-variables}{%
\subsection{Continious Variables}\label{continious-variables}}

\hypertarget{split-group-stats-continious}{%
\subsubsection{Split-Group Stats
Continious}\label{split-group-stats-continious}}

\hypertarget{grouped-continious}{%
\subsubsection{Grouped Continious}\label{grouped-continious}}

\pagebreak

\newpage
\begin{landscape}

\hypertarget{cross-tables}{%
\section{Cross Tables}\label{cross-tables}}

\end{landscape}

\hypertarget{plots}{%
\section{Plots}\label{plots}}

\hypertarget{categorical-variables-1}{%
\subsection{Categorical Variables}\label{categorical-variables-1}}

\hypertarget{plots-1}{%
\section{Plots}\label{plots-1}}

\hypertarget{continious-variables-1}{%
\subsection{Continious Variables}\label{continious-variables-1}}

\newpage
\begin{landscape}

\hypertarget{survival-analysis}{%
\section{Survival Analysis}\label{survival-analysis}}

\hypertarget{pairwise-comparison}{%
\section{Pairwise comparison}\label{pairwise-comparison}}

\hypertarget{multivariate-analysis-survival}{%
\section{Multivariate Analysis
Survival}\label{multivariate-analysis-survival}}

\begin{center}\rule{0.5\linewidth}{\linethickness}\end{center}

\begin{center}\rule{0.5\linewidth}{\linethickness}\end{center}

\hypertarget{km-plot}{%
\section{KM plot}\label{km-plot}}

explanatory = c(``perfor.factor'') dependent = ``Surv(time, status)''
colon\_s \%\textgreater\% surv.plot(dependent, explanatory, xlab=``Time
(days)'', pval=TRUE, legend=``none'')

Notes

Use Hmisc::label() to assign labels to variables for tables and plots.

label(colon\_s\$age.factor) = ``Age (years)''

Export dataframe tables directly or to R Markdown using knitr::kable().

Note wrapper summary.missing() can be useful. Wraps mice::md.pattern.

colon\_s \%\textgreater\% summary.missing(dependent, explanatory)

\end{landscape}

Where a multivariable model contains a subset of the variables specified
in the full univariable set, this can be specified.

explanatory = c(``age.factor'', ``sex.factor'', ``obstruct.factor'',
``perfor.factor'') explanatory.multi = c(``age.factor'',
``obstruct.factor'') dependent = `mort\_5yr' colon\_s \%\textgreater\%
summarizer(dependent, explanatory, explanatory.multi)

Random effects.

e.g.~lme4::glmer(dependent \textasciitilde{} explanatory + (1 \textbar{}
random\_effect), family=``binomial'')

explanatory = c(``age.factor'', ``sex.factor'', ``obstruct.factor'',
``perfor.factor'') explanatory.multi = c(``age.factor'',
``obstruct.factor'') random.effect = ``hospital'' dependent =
`mort\_5yr' colon\_s \%\textgreater\% summarizer(dependent, explanatory,
explanatory.multi, random.effect)

metrics=TRUE provides common model metrics.

colon\_s \%\textgreater\% summarizer(dependent, explanatory,
explanatory.multi, metrics=TRUE)

Cox proportional hazards

e.g.~survival::coxph(dependent \textasciitilde{} explanatory)

explanatory = c(``age.factor'', ``sex.factor'', ``obstruct.factor'',
``perfor.factor'') dependent = ``Surv(time, status)''

colon\_s \%\textgreater\% summarizer(dependent, explanatory)

Rather than going all-in-one, any number of subset models can be
manually added on to a summary.factorlist() table using
summarizer.merge(). This is particularly useful when models take a
long-time to run or are complicated.

Note requirement for glm.id=TRUE. fit2df is a subfunction extracting
most common models to a dataframe.

explanatory = c(``age.factor'', ``sex.factor'', ``obstruct.factor'',
``perfor.factor'') explanatory.multi = c(``age.factor'',
``obstruct.factor'') random.effect = ``hospital'' dependent =
`mort\_5yr'

\hypertarget{separate-tables}{%
\section{Separate tables}\label{separate-tables}}

colon\_s \%\textgreater\% summary.factorlist(dependent, explanatory,
glm.id=TRUE) -\textgreater{} example.summary

colon\_s \%\textgreater\% glmuni(dependent, explanatory)
\%\textgreater\% fit2df(estimate.suffix=" (univariable)")
-\textgreater{} example.univariable

colon\_s \%\textgreater\% glmmulti(dependent, explanatory)
\%\textgreater\% fit2df(estimate.suffix=" (multivariable)")
-\textgreater{} example.multivariable

colon\_s \%\textgreater\% glmmixed(dependent, explanatory,
random.effect) \%\textgreater\% fit2df(estimate.suffix=" (multilevel")
-\textgreater{} example.multilevel

\hypertarget{pipe-together}{%
\section{Pipe together}\label{pipe-together}}

example.summary \%\textgreater\% summarizer.merge(example.univariable)
\%\textgreater\% summarizer.merge(example.multivariable)
\%\textgreater\% summarizer.merge(example.multilevel) \%\textgreater\%
select(-c(glm.id, index)) -\textgreater{} example.final example.final

Cox Proportional Hazards example with separate tables merged together.

explanatory = c(``age.factor'', ``sex.factor'', ``obstruct.factor'',
``perfor.factor'') explanatory.multi = c(``age.factor'',
``obstruct.factor'') dependent = ``Surv(time, status)''

\hypertarget{separate-tables-1}{%
\section{Separate tables}\label{separate-tables-1}}

colon\_s \%\textgreater\% summary.factorlist(dependent, explanatory,
glm.id=TRUE) -\textgreater{} example2.summary

colon\_s \%\textgreater\% coxphuni(dependent, explanatory)
\%\textgreater\% fit2df(estimate.suffix=" (univariable)")
-\textgreater{} example2.univariable

colon\_s \%\textgreater\% coxphmulti(dependent, explanatory.multi)
\%\textgreater\% fit2df(estimate.suffix=" (multivariable)")
-\textgreater{} example2.multivariable

\hypertarget{pipe-together-1}{%
\section{Pipe together}\label{pipe-together-1}}

example2.summary \%\textgreater\% summarizer.merge(example2.univariable)
\%\textgreater\% summarizer.merge(example2.multivariable)
\%\textgreater\% select(-c(glm.id, index)) -\textgreater{}
example2.final example2.final

\hypertarget{or-plot}{%
\section{OR plot}\label{or-plot}}

explanatory = c(``age.factor'', ``sex.factor'', ``obstruct.factor'',
``perfor.factor'') dependent = `mort\_5yr' colon\_s \%\textgreater\%
or.plot(dependent, explanatory) \# Previously fitted models
(\texttt{glmmulti()} or \texttt{glmmixed()}) can be provided directly to
\texttt{glmfit}

\hypertarget{hr-plot-not-fully-tested}{%
\section{HR plot (not fully tested)}\label{hr-plot-not-fully-tested}}

explanatory = c(``age.factor'', ``sex.factor'', ``obstruct.factor'',
``perfor.factor'') dependent = ``Surv(time, status)'' colon\_s
\%\textgreater\% hr.plot(dependent, explanatory, dependent\_label =
``Survival'') \# Previously fitted models (\texttt{coxphmulti}) can be
provided directly using \texttt{coxfit}

\hypertarget{anova}{%
\section{ANOVA}\label{anova}}

\hypertarget{save-final-data}{%
\section{Save Final Data}\label{save-final-data}}

\pagebreak

\hypertarget{final-data-summary}{%
\section{Final Data Summary}\label{final-data-summary}}

\pagebreak

\hypertarget{software-and-libraries-used}{%
\section{Software and Libraries
Used}\label{software-and-libraries-used}}

The jamovi project (2019). jamovi. (Version 0.9) {[}Computer
Software{]}. Retrieved from \url{https://www.jamovi.org}. R Core Team
(2018). R: A Language and envionment for statistical computing.
{[}Computer software{]}. Retrieved from
\url{https://cran.r-project.org/}. Fox, J., \& Weisberg, S. (2018). car:
Companion to Applied Regression. {[}R package{]}. Retrieved from
\url{https://cran.r-project.org/package=car}.

\pagebreak

\hypertarget{session-info}{%
\section{Session Info}\label{session-info}}

\pagebreak

\hypertarget{notes}{%
\section{Notes}\label{notes}}

Last update on 2019-11-06 12:29:44

Serdar Balci, MD, Pathologist\\
\href{mailto:drserdarbalci@gmail.com}{\nolinkurl{drserdarbalci@gmail.com}}\\
\url{https://rpubs.com/sbalci/CV}

\pagebreak

\newpage


\end{document}
