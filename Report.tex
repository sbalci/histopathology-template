\documentclass[]{article}
\usepackage{lmodern}
\usepackage{amssymb,amsmath}
\usepackage{ifxetex,ifluatex}
\usepackage{fixltx2e} % provides \textsubscript
\ifnum 0\ifxetex 1\fi\ifluatex 1\fi=0 % if pdftex
  \usepackage[T1]{fontenc}
  \usepackage[utf8]{inputenc}
\else % if luatex or xelatex
  \ifxetex
    \usepackage{mathspec}
  \else
    \usepackage{fontspec}
  \fi
  \defaultfontfeatures{Ligatures=TeX,Scale=MatchLowercase}
\fi
% use upquote if available, for straight quotes in verbatim environments
\IfFileExists{upquote.sty}{\usepackage{upquote}}{}
% use microtype if available
\IfFileExists{microtype.sty}{%
\usepackage{microtype}
\UseMicrotypeSet[protrusion]{basicmath} % disable protrusion for tt fonts
}{}
\usepackage[margin=1in]{geometry}
\usepackage{hyperref}
\hypersetup{unicode=true,
            pdftitle={Histopathology Research Template},
            pdfborder={0 0 0},
            breaklinks=true}
\urlstyle{same}  % don't use monospace font for urls
\usepackage{color}
\usepackage{fancyvrb}
\newcommand{\VerbBar}{|}
\newcommand{\VERB}{\Verb[commandchars=\\\{\}]}
\DefineVerbatimEnvironment{Highlighting}{Verbatim}{commandchars=\\\{\}}
% Add ',fontsize=\small' for more characters per line
\usepackage{framed}
\definecolor{shadecolor}{RGB}{255,255,255}
\newenvironment{Shaded}{\begin{snugshade}}{\end{snugshade}}
\newcommand{\AlertTok}[1]{\textcolor[rgb]{0.75,0.01,0.01}{\textbf{\colorbox[rgb]{0.97,0.90,0.90}{#1}}}}
\newcommand{\AnnotationTok}[1]{\textcolor[rgb]{0.79,0.38,0.79}{#1}}
\newcommand{\AttributeTok}[1]{\textcolor[rgb]{0.00,0.34,0.68}{#1}}
\newcommand{\BaseNTok}[1]{\textcolor[rgb]{0.69,0.50,0.00}{#1}}
\newcommand{\BuiltInTok}[1]{\textcolor[rgb]{0.39,0.29,0.61}{\textbf{#1}}}
\newcommand{\CharTok}[1]{\textcolor[rgb]{0.57,0.30,0.62}{#1}}
\newcommand{\CommentTok}[1]{\textcolor[rgb]{0.54,0.53,0.53}{#1}}
\newcommand{\CommentVarTok}[1]{\textcolor[rgb]{0.00,0.58,1.00}{#1}}
\newcommand{\ConstantTok}[1]{\textcolor[rgb]{0.67,0.33,0.00}{#1}}
\newcommand{\ControlFlowTok}[1]{\textcolor[rgb]{0.12,0.11,0.11}{\textbf{#1}}}
\newcommand{\DataTypeTok}[1]{\textcolor[rgb]{0.00,0.34,0.68}{#1}}
\newcommand{\DecValTok}[1]{\textcolor[rgb]{0.69,0.50,0.00}{#1}}
\newcommand{\DocumentationTok}[1]{\textcolor[rgb]{0.38,0.47,0.50}{#1}}
\newcommand{\ErrorTok}[1]{\textcolor[rgb]{0.75,0.01,0.01}{\underline{#1}}}
\newcommand{\ExtensionTok}[1]{\textcolor[rgb]{0.00,0.58,1.00}{\textbf{#1}}}
\newcommand{\FloatTok}[1]{\textcolor[rgb]{0.69,0.50,0.00}{#1}}
\newcommand{\FunctionTok}[1]{\textcolor[rgb]{0.39,0.29,0.61}{#1}}
\newcommand{\ImportTok}[1]{\textcolor[rgb]{1.00,0.33,0.00}{#1}}
\newcommand{\InformationTok}[1]{\textcolor[rgb]{0.69,0.50,0.00}{#1}}
\newcommand{\KeywordTok}[1]{\textcolor[rgb]{0.12,0.11,0.11}{\textbf{#1}}}
\newcommand{\NormalTok}[1]{\textcolor[rgb]{0.12,0.11,0.11}{#1}}
\newcommand{\OperatorTok}[1]{\textcolor[rgb]{0.12,0.11,0.11}{#1}}
\newcommand{\OtherTok}[1]{\textcolor[rgb]{0.00,0.43,0.16}{#1}}
\newcommand{\PreprocessorTok}[1]{\textcolor[rgb]{0.00,0.43,0.16}{#1}}
\newcommand{\RegionMarkerTok}[1]{\textcolor[rgb]{0.00,0.34,0.68}{\colorbox[rgb]{0.88,0.91,0.97}{#1}}}
\newcommand{\SpecialCharTok}[1]{\textcolor[rgb]{0.24,0.68,0.91}{#1}}
\newcommand{\SpecialStringTok}[1]{\textcolor[rgb]{1.00,0.33,0.00}{#1}}
\newcommand{\StringTok}[1]{\textcolor[rgb]{0.75,0.01,0.01}{#1}}
\newcommand{\VariableTok}[1]{\textcolor[rgb]{0.00,0.34,0.68}{#1}}
\newcommand{\VerbatimStringTok}[1]{\textcolor[rgb]{0.75,0.01,0.01}{#1}}
\newcommand{\WarningTok}[1]{\textcolor[rgb]{0.75,0.01,0.01}{#1}}
\usepackage{graphicx,grffile}
\makeatletter
\def\maxwidth{\ifdim\Gin@nat@width>\linewidth\linewidth\else\Gin@nat@width\fi}
\def\maxheight{\ifdim\Gin@nat@height>\textheight\textheight\else\Gin@nat@height\fi}
\makeatother
% Scale images if necessary, so that they will not overflow the page
% margins by default, and it is still possible to overwrite the defaults
% using explicit options in \includegraphics[width, height, ...]{}
\setkeys{Gin}{width=\maxwidth,height=\maxheight,keepaspectratio}
\IfFileExists{parskip.sty}{%
\usepackage{parskip}
}{% else
\setlength{\parindent}{0pt}
\setlength{\parskip}{6pt plus 2pt minus 1pt}
}
\setlength{\emergencystretch}{3em}  % prevent overfull lines
\providecommand{\tightlist}{%
  \setlength{\itemsep}{0pt}\setlength{\parskip}{0pt}}
\setcounter{secnumdepth}{5}
% Redefines (sub)paragraphs to behave more like sections
\ifx\paragraph\undefined\else
\let\oldparagraph\paragraph
\renewcommand{\paragraph}[1]{\oldparagraph{#1}\mbox{}}
\fi
\ifx\subparagraph\undefined\else
\let\oldsubparagraph\subparagraph
\renewcommand{\subparagraph}[1]{\oldsubparagraph{#1}\mbox{}}
\fi

%%% Use protect on footnotes to avoid problems with footnotes in titles
\let\rmarkdownfootnote\footnote%
\def\footnote{\protect\rmarkdownfootnote}

%%% Change title format to be more compact
\usepackage{titling}

% Create subtitle command for use in maketitle
\providecommand{\subtitle}[1]{
  \posttitle{
    \begin{center}\large#1\end{center}
    }
}

\setlength{\droptitle}{-2em}

  \title{Histopathology Research Template}
    \pretitle{\vspace{\droptitle}\centering\huge}
  \posttitle{\par}
    \author{true}
    \preauthor{\centering\large\emph}
  \postauthor{\par}
      \predate{\centering\large\emph}
  \postdate{\par}
    \date{2019-11-09}

\usepackage{pdflscape}
\newcommand{\blandscape}{\begin{landscape}}
\newcommand{\elandscape}{\end{landscape}}
\usepackage{xcolor}
\usepackage{afterpage}

\begin{document}
\maketitle

{
\setcounter{tocdepth}{5}
\tableofcontents
}
\hypertarget{header-codes}{%
\subsection{Header Codes}\label{header-codes}}

\textbf{Codes for general settings.}\footnote{See
  \texttt{childRmd/\_header.Rmd} file for other general settings}

\textbf{setup global chunk settings}

Change \texttt{echo\ =\ FALSE} to hide codes after knitting.

\begin{Shaded}
\begin{Highlighting}[]
\NormalTok{knitr}\OperatorTok{::}\NormalTok{opts_chunk}\OperatorTok{$}\KeywordTok{set}\NormalTok{(}
    \DataTypeTok{eval =} \OtherTok{TRUE}\NormalTok{,}
    \DataTypeTok{echo =} \OtherTok{TRUE}\NormalTok{,}
    \DataTypeTok{fig.path =}\NormalTok{ here}\OperatorTok{::}\KeywordTok{here}\NormalTok{(}\StringTok{"figs/"}\NormalTok{),}
    \DataTypeTok{message =} \OtherTok{FALSE}\NormalTok{,}
    \DataTypeTok{warning =} \OtherTok{FALSE}\NormalTok{,}
    \DataTypeTok{error =} \OtherTok{FALSE}\NormalTok{,}
    \DataTypeTok{cache =} \OtherTok{FALSE}\NormalTok{,}
    \DataTypeTok{comment =} \OtherTok{NA}\NormalTok{,}
    \DataTypeTok{tidy =} \OtherTok{TRUE}\NormalTok{,}
    \DataTypeTok{fig.width =} \DecValTok{12}\NormalTok{,}
    \DataTypeTok{fig.height =} \DecValTok{8}
\NormalTok{)}
\end{Highlighting}
\end{Shaded}

\textbf{load library}

see \texttt{R/loadLibrary.R} for the libraries loaded.

\begin{Shaded}
\begin{Highlighting}[]
\KeywordTok{source}\NormalTok{(}\DataTypeTok{file =}\NormalTok{ here}\OperatorTok{::}\KeywordTok{here}\NormalTok{(}\StringTok{"R"}\NormalTok{, }\StringTok{"loadLibrary.R"}\NormalTok{))}
\end{Highlighting}
\end{Shaded}

\begin{center}\rule{0.5\linewidth}{\linethickness}\end{center}

\hypertarget{import-data}{%
\subsection{Import Data}\label{import-data}}

\textbf{Codes for importing data.}\footnote{See
  \texttt{childRmd/\_importData.Rmd} file for other codes}

\textbf{Use this code to generate fake clinicopathologic data}

\begin{Shaded}
\begin{Highlighting}[]
\KeywordTok{source}\NormalTok{(}\DataTypeTok{file =}\NormalTok{ here}\OperatorTok{::}\KeywordTok{here}\NormalTok{(}\StringTok{"R"}\NormalTok{, }\StringTok{"generatingFakeData.R"}\NormalTok{))}
\end{Highlighting}
\end{Shaded}

\textbf{Use this code to read the generated fake data}

\begin{Shaded}
\begin{Highlighting}[]
\KeywordTok{library}\NormalTok{(readxl)}
\NormalTok{mydata <-}\StringTok{ }\NormalTok{readxl}\OperatorTok{::}\KeywordTok{read_excel}\NormalTok{(here}\OperatorTok{::}\KeywordTok{here}\NormalTok{(}\StringTok{"data"}\NormalTok{, }\StringTok{"mydata.xlsx"}\NormalTok{))}
\CommentTok{# View(mydata)}
\end{Highlighting}
\end{Shaded}

\begin{center}\rule{0.5\linewidth}{\linethickness}\end{center}


\end{document}
